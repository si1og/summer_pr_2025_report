\section*{Приложение Б. Задание 1. Реализация функций библиотеки поиска максимально элемента}
\addcontentsline{toc}{section}{Приложение Б. Задание 1. Реализация функций библиотеки поиска максимально элемента}

\begin{lstlisting}[language=C,numbers=left]
#include "max.h"

int max_iterative(const int *arr, int size) {
  if (size <= 0) return 0;
  int max = arr[0];
  for (int i = 1; i < size; ++i) {
    if (arr[i] > max)
      max = arr[i];
  }
  return max;
}

int max_recursive(const int *arr, int size) {
  if (size <= 0) return 0;
  if (size == 1)
    return arr[0];
  int m = max_recursive(arr, size - 1);
  return (arr[size - 1] > m) ? arr[size - 1] : m;
}

double max_iterative_double(const double *arr, int size) {
  if (size <= 0) return 0.0;
  double max = arr[0];
  for (int i = 1; i < size; ++i) {
    if (arr[i] > max)
      max = arr[i];
  }
  return max;
}

double max_recursive_double(const double *arr, int size) {
  if (size <= 0) return 0.0;
  if (size == 1)
    return arr[0];
  double m = max_recursive_double(arr, size - 1);
  return (arr[size - 1] > m) ? arr[size - 1] : m;
}

char max_iterative_char(const char *arr, int size) {
  if (size <= 0) return '\0';
  char max = arr[0];
  for (int i = 1; i < size; ++i) {
    if (arr[i] > max)
      max = arr[i];
  }
  return max;
}

char max_recursive_char(const char *arr, int size) {
  if (size <= 0) return '\0';
  if (size == 1)
    return arr[0];
  char m = max_recursive_char(arr, size - 1);
  return (arr[size - 1] > m) ? arr[size - 1] : m;
}
\end{lstlisting}