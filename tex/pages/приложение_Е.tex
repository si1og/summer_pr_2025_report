\section*{Приложение Е. Задание 2. Реализация функций работы с псевдографическим интерфейсом и базой данных}
\addcontentsline{toc}{section}{Приложение Е. Задание 2. Реализация функций работы с псевдографическим интерфейсом и базой данных}

\begin{lstlisting}[language=C,numbers=left]
#include <string.h>
#include <stdio.h>
#include <stdlib.h>
#include <ncurses.h>
#include <zlib.h>
#include "app.h"

void input_znak(ZNAK *z) {
  echo();
  while (1) {
    mvprintw(2, 2, "Фамилия и инициалы: ");
    getnstr(z->name, sizeof(z->name) - 1);
    if (strlen(z->name) > 0) break;
    error_msg("Фамилия и инициалы не могут быть пустыми.");
  }

  while (1) {
    mvprintw(3, 2, "Знак зодиака: ");
    getnstr(z->zodiac, sizeof(z->zodiac) - 1);
    if (strlen(z->zodiac) > 0) break;
    error_msg("Знак зодиака не может быть пустым.");
  }

  while (1) {
    int birth_year = 0;
    mvprintw(4, 2, "Год рождения: ");
    scanw("%d", &birth_year);
    if (birth_year >= 1899 && birth_year <= 2025) {
      z->birth_year = birth_year;
      break;
    } else {
      error_msg("Год рождения должен быть в диапазоне 1899-2025.");
    }
  }
  noecho();
}

void save_db(ZNAK_DB_HEADER *header, ZNAK *arr, int count) {
  FILE *file = fopen(DB_FILENAME, "wb");
  if (!file) {
    perror("Ошибка создания файла");
    return;
  }

  header->count = count;
  header->transaction_id++;
  header->checksum = crc32(0L, Z_NULL, 0);
  header->checksum = crc32(header->checksum, (const Bytef*)arr, sizeof(ZNAK) * count);

  fwrite(header, sizeof(ZNAK_DB_HEADER), 1, file);
  fwrite(arr, sizeof(ZNAK), count, file);
  fclose(file);
}

void find_by_year_and_zodiac(ZNAK_DB_HEADER *header, 
  ZNAK *arr, int count, const char *zodiac, int year) {
  save_db(header, arr, count);

  int found = 0;

  ZNAK found_arr[count];
  for (int i = 0; i < count; ++i) {
    if (arr[i].birth_year == year && \
      strcmp(arr[i].zodiac, zodiac) == 0) {
      found_arr[found++] = arr[i];
    }
  }
  
  clear();
  mvprintw(1, 2, "Результаты поиска по знаку \"%s\" и году %d:",
          zodiac, year);

  if (found) {
    db_print_interface(header, found_arr, found);
  } else {
    mvprintw(5, 4, "Совпадений не найдено.");
    mvprintw(20, 2, "Нажмите любую клавишу для выхода...");
    getch();
  }
}

void add_record(ZNAK_DB_HEADER *header, ZNAK *arr, int *count) {
  clear();
  mvprintw(1, 2, "Добавление новой записи:");
  input_znak(&arr[*count]);
  (*count)++;
  save_db(header, arr, *count);
}

void search_records(ZNAK_DB_HEADER *header, ZNAK *arr, int count) {
  char search_zodiac[ZODIAC_LEN];
  int search_year;

  clear();
  mvprintw(1, 2, "Введите знак зодиака: ");
  echo();
  getnstr(search_zodiac, sizeof(search_zodiac) - 1);
  noecho();

  mvprintw(2, 2, "Введите год рождения: ");
  echo();
  scanw("%d", &search_year);
  noecho();

  find_by_year_and_zodiac(header, arr, count, search_zodiac,
  search_year);
}

void print_db_header(ZNAK_DB_HEADER *header) {
  clear();
  mvprintw(1, 2, "Заголовок базы данных:");
  mvprintw(2, 4, "Сигнатура: %.4s", header->signature);
  mvprintw(3, 4, "Номер транзакции: %u", header->transaction_id);
  mvprintw(4, 4, "Количество записей: %u", header->count);
  mvprintw(5, 4, "Контрольная сумма (CRC32): %u", header->checksum);

  mvprintw(20, 2, "Нажмите любую клавишу для выхода...");
  getch();
}

void print_db(ZNAK_DB_HEADER *header, ZNAK *arr, int count) {
  clear();
  save_db(header, arr, count);
  mvprintw(1, 2, "Содержимое базы данных:");
  db_print_interface(header, arr, count);
}

void clear_except_first_line() {
  int rows, cols;
  getmaxyx(stdscr, rows, cols);

  for (int i = 2; i <= rows; ++i) {
    move(i, 0);
    clrtoeol();
  }
}

void db_print_interface(ZNAK_DB_HEADER *header, ZNAK *arr, int count) {
  int rows, cols;
  getmaxyx(stdscr, rows, cols);
  
  int lines_per_page = rows - 10;
  int start = 0;
  
  while (start < count) {
    clear_except_first_line();
    mvprintw(2, 4, "%-7s %-41s %-26s %-5s", "№", 
    "Фамилия и инициалы", "Знак зодиака", "Год");
    mvhline(3, 4, '-', cols - 8);

    int i, row = 4;
    for (i = start; i < count && i < start + lines_per_page; ++i) {
      mvprintw(row++, 4, "%-5d %-25s %-15s %-5d",
                i + 1, arr[i].name, arr[i].zodiac, arr[i].birth_year);
    }

    mvprintw(rows - 2, 2,
"Страница %d/%d. Нажмите 'n' для следующей, 'p' для 
предыдущей, 'q' для выхода.",
      (start / lines_per_page) + 1,
      (count + lines_per_page - 1) / lines_per_page);

    int ch = getch();
    if (ch == 'q' || ch == 'Q')
      break;
    else if (ch == 'n' || ch == 'N')
      start += lines_per_page;
    else if (ch == 'p' || ch == 'P') {
      if (start - lines_per_page >= 0)
        start -= lines_per_page;
    } else
      break;
  }
}

void error_msg(const char *msg) {
  clear();
  mvprintw(1, 2, msg);
  mvprintw(3, 2, "Нажмите любую клавишу для повторной попытки...");
  getch();
  clear();
}
\end{lstlisting}