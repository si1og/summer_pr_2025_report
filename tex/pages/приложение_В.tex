\section*{Приложение В. Задание 1. Реализация тестирующей функций библиотеки поиска максимально элемента}
\addcontentsline{toc}{section}{Приложение В. Задание 1. Реализация тестирующей функций библиотеки поиска максимально элемента}

\begin{lstlisting}[language=C,numbers=left]
// test.c

#include <stdio.h>
#include <sys/time.h>
#include "max.h"

unsigned int seed = 123456789;
int myrand() {
  seed = seed * 1103515245 + 12345;
  return (seed >> 16) & 0x7FFF;
}

double measure(int (*func)(const int*, int), const int* arr, int size) {
  struct timeval start, end;
  gettimeofday(&start, NULL);
  int result = func(arr, size);
  gettimeofday(&end, NULL);

  double elapsed = (end.tv_sec - start.tv_sec) * 1000.0;
  elapsed += (end.tv_usec - start.tv_usec) / 1000.0;

  printf("Result: %d, Time: %.3f ms\n", result, elapsed);
  return elapsed;
}

int main() {
  const int SIZE = 100000;
  int arr[SIZE];

  for (int i = 0; i < SIZE; ++i)
      arr[i] = myrand();

  printf("Iterative max:\n");
  measure(max_iterative, arr, SIZE);

  printf("Recursive max:\n");
  measure(max_recursive, arr, SIZE);

  return 0;
}

// test_dlopen.c

#include <stdio.h>
#include <stdlib.h>
#include <dlfcn.h>
#include <sys/time.h>

unsigned int seed = 123456789;
int myrand() {
  seed = seed * 1103515245 + 12345;
  return (seed >> 16) & 0x7FFF;
}

double measure(int (*func)(const int*, int), const int* arr, int size) {
  struct timeval start, end;
  gettimeofday(&start, NULL);
  int result = func(arr, size);
  gettimeofday(&end, NULL);

  double elapsed = (end.tv_sec - start.tv_sec) * 1000.0;
  elapsed += (end.tv_usec - start.tv_usec) / 1000.0;

  printf("Result: %d, Time: %.3f ms\n", result, elapsed);
  return elapsed;
}

int main() {
  void *handle;
  int (*max_iterative)(const int*, int);
  int (*max_recursive)(const int*, int);
  char *error;

  // Открываем библиотеку
  handle = dlopen("./libmax.so", RTLD_LAZY);
  if (!handle) {
    fprintf(stderr, "Ошибка загрузки: %s\n", dlerror());
    exit(EXIT_FAILURE);
  }

  // Получаем адреса функций
  max_iterative = (int (*)(const int*, int)) dlsym(handle, "max_iterative");
  if ((error = dlerror()) != NULL)  {
    fprintf(stderr, "Ошибка поиска max_iterative: %s\n", error);
    exit(EXIT_FAILURE);
  }

  max_recursive = (int (*)(const int*, int)) dlsym(handle, "max_recursive");
  if ((error = dlerror()) != NULL)  {
    fprintf(stderr, "Ошибка поиска max_recursive: %s\n", error);
    exit(EXIT_FAILURE);
  }

  // Заполняем массив и тестируем
  const int SIZE = 100000;
  int arr[SIZE];
  for (int i = 0; i < SIZE; ++i)
      arr[i] = myrand();

  printf("Iterative max via dlopen:\n");
  measure(max_iterative, arr, SIZE);

  printf("Recursive max via dlopen:\n");
  measure(max_recursive, arr, SIZE);

  dlclose(handle);
  return 0;
}

\end{lstlisting}