\section{Постановка задачи}

В рамках данной работы требуется реализовать и протестировать комплекс программ, демонстрирующих работу со статическими и динамическими библиотеками, а также разработать приложение для управления бинарной базой данных с визуализацией в псевдографике на основе библиотеки \texttt{ncurses} в среде GNU/Linux.

\subsubsection*{Исходные данные.}
\begin{itemize}
\item В качестве среды разработки используется виртуальная машина VirtualBox с установленным дистрибутивом Debian 12 под архитектуру \texttt{arm64} (Apple Silicon, Apple M3), настроенная для работы без графической оболочки (ранее уже оговаривалось во введении).
\item Используется компилятор \texttt{gcc}, системы сборки \texttt{cmake} и инструменты анализа зависимостей и бинарных данных (\texttt{ldd}, \texttt{hexdump}).
\item Для интерфейса программы использован пакет \texttt{ncurses}, позволяющий создавать текстовые интерфейсы в терминале.
\item Контроль целостности данных реализуется с использованием CRC-32 из библиотеки \texttt{zlib}.
\end{itemize}

\subsubsection*{При выполнении работы необходимо:}
\begin{itemize}
\item Разработать библиотеку нахождения максимального элемента в массиве, реализованную двумя способами — с рекурсией и без, без использования библиотечных функций. Оформить её в виде статической и динамической библиотек.
\item Подготовить тестовую программу, сравнивающую производительность работы библиотеки при различных вариантах линковки (статической и динамической), а также при подгрузке библиотеки во время выполнения через \texttt{dlopen}. Провести замеры времени исполнения с помощью \texttt{time} для разных уровней оптимизации компилятора (-O0, -O1, -O2, -Os), и сделать выводы о влиянии оптимизаций.
\item Исследовать зависимости скомпилированных программ от разделяемых библиотек с помощью команды \texttt{ldd} и визуализировать полученные зависимости в виде рекурсивного дерева.
\item Описать структуру \texttt{ZNAK}, содержащую следующие поля:
\begin{itemize}
\item фамилия и инициалы (строка языка Си),
\item знак зодиака (строка языка Си),
\item год рождения (целое число).
\end{itemize}
\item Разработать бинарный формат хранения базы данных, включающий заголовок с полями:
\begin{itemize}
\item сигнатура (4 байта) — первые четыре буквы фамилии латиницей,
\item номер транзакции (4 байта), увеличивающийся при каждом чтении и записи,
\item количество структур \texttt{ZNAK} (4 байта),
\item контрольная сумма CRC-32 (4 байта), вычисляемая для данных, следующих за заголовком.
\end{itemize}
\item Реализовать программу с использованием библиотеки \texttt{ncurses}, выполняющую:
\begin{itemize}
\item создание нового бинарного файла базы данных при отсутствии файла, с вводом массива структур \texttt{ZNAK} через псевдографический интерфейс,
\item поиск и вывод информации о людях, родившихся в указанный год с заданным знаком зодиака (или сообщение об отсутствии таких записей),
\item добавление новой записи и обновление заголовка с корректировкой CRC-32.
\end{itemize}
\item Зафиксировать промежуточные состояния бинарного файла (например, с помощью \texttt{hexdump}), продемонстрировав изменения заголовка после операций с файлом.
\item Провести анализ зависимости итоговой программы от динамических библиотек через \texttt{ldd} и построить соответствующее рекурсивное дерево.
\end{itemize}

\subsubsection*{Ожидаемые результаты.}
В результате выполнения работы должны быть получены:
\begin{itemize}
\item Статическая и динамическая библиотеки с функцией поиска максимума, а также отчёт о сравнении их производительности.
\item Исполняемая программа для управления базой данных в формате бинарного файла с заголовком и массивом структур \texttt{ZNAK}.
\item Подробный отчёт с демонстрацией работы программы в текстовом интерфейсе (скриншоты), выводами утилит \texttt{ldd}, \texttt{hexdump}, и схемой рекурсивного дерева библиотечных зависимостей.
\end{itemize}