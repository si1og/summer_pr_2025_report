\section*{Заключение}
\addcontentsline{toc}{section}{Заключение}

В ходе выполнения данной работы были последовательно реализованы все задачи, сформулированные в постановке. В рамках первого задания была написана библиотека с функцией поиска максимального элемента в массиве целых чисел, реализованной как с использованием рекурсии, так и итеративно, собранной в виде как статической (\texttt{.a}), так и динамической (\texttt{.so}) библиотек. Для тестирования скорости работы функции были подготовлены специальные программы с различными режимами сборки и выполнены эксперименты с флагами оптимизации компилятора \texttt{-O0}, \texttt{-O1}, \texttt{-O2} и \texttt{-Os}, что позволило на практике оценить влияние оптимизаций на производительность.

Также была разработана версия программы с использованием механизма подгружаемых библиотек через \texttt{dlopen}, что дало возможность изучить отличие явной динамической загрузки функций от традиционной динамической линковки на этапе запуска программы. Проведённые тесты показали, что такой подход работает, но имеет некоторые накладные расходы на вызовы через полученные указатели.

Во втором задании была описана структура \texttt{ZNAK}, а также реализована программа с использованием библиотеки \texttt{ncurses}, позволяющая в текстовом интерфейсе создавать и редактировать бинарную базу данных с заголовком, содержащим сигнатуру, номер транзакции, количество записей и контрольную сумму по CRC-32. Программа поддерживает ввод массива структур с клавиатуры, поиск людей по знаку зодиака и году рождения, а также добавление новых записей с корректной перерасчёткой заголовка базы. В качестве дополнительного исследования были использованы утилиты \texttt{hexdump} для просмотра бинарного содержимого файлов и \texttt{ldd} для анализа зависимостей программ, с построением дерева библиотек.

В результате были освоены и применены на практике следующие технологии и инструменты: работа в терминале GNU/Linux без графической оболочки, настройка гостевой системы в VirtualBox, сборка проектов с помощью \texttt{gcc} и \texttt{cmake}, статическая и динамическая линковка, использование \texttt{dlopen}, а также создание интерфейсов в псевдографике через \texttt{ncurses}. Помимо этого, удалось лучше понять механизм загрузки разделяемых библиотек в Linux. Все пункты, указанные в постановке задачи, были выполнены полностью.
