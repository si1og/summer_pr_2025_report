\section*{Заключение}
\addcontentsline{toc}{section}{Заключение}

В ходе выполнения работы были выполнены все задачи, поставленные в разделе постановки задачи, с использованием современных средств разработки на платформе GNU/Linux в виртуальной среде VirtualBox. Подробно опишем каждый пункт:

\begin{itemize}
    \item[1.] Разработана библиотека на языке Си для нахождения максимального элемента в массиве, содержащая две реализации: рекурсивную и итеративную. В реализации не использовались библиотечные функции.

    \item[2.1.] Полученные исходные файлы библиотеки были собраны в два варианта:
    \begin{itemize}
        \item статическая библиотека (\texttt{.a}), компонуемая на этапе сборки вместе с программой,
        \item динамическая библиотека (\texttt{.so}), подключаемая при запуске программы.
    \end{itemize}
    Для сборки использовалась система \texttt{cmake}, что упростило процесс компиляции и управления флагами компилятора.

    \item[2.2.] Создана тестовая программа на Си, измеряющая время выполнения функций поиска максимального элемента при различных вариантах использования библиотеки (статическая, динамическая линковка). Для измерений времени использовались функции \texttt{gettimeofday} и вывод результатов в миллисекундах. 

    \item[2.3.] Программа была дополнена вариантами компиляции с разными уровнями оптимизации компилятора (\texttt{-O0}, \texttt{-O1}, \texttt{-O2}, \texttt{-O3}, \texttt{-Os}). Для автоматизации сборки с различными флагами использован скрипт на \texttt{bash}. Проведены сравнительные эксперименты, показавшие значительное ускорение выполнения кода при включении оптимизаций, особенно для итеративного варианта.

    \item[2.4.] Для анализа работы динамической загрузки функций написан отдельный тест, в котором функции \texttt{max\_iterative} и \texttt{max\_recursive} подгружались во время выполнения через \texttt{dlopen} и \texttt{dlsym}. Это позволило оценить плюсы (гибкость загрузки) и минусы (потенциальные накладные расходы и необходимость явного контроля ошибок загрузки) подхода по сравнению с обычной динамической линковкой.

    \item[3.] С помощью утилиты \texttt{ldd} были исследованы зависимости скомпилированных исполняемых файлов от системных разделяемых библиотек. Результаты использовались для построения рекурсивного дерева зависимостей в отчёте, что позволило визуально увидеть, какие библиотеки требуются для работы программы.

    \item[4-5.] Реализована структура данных \texttt{ZNAK}, включающая фамилию и инициалы (строка), знак зодиака (строка) и год рождения (целое число). Также определён заголовок бинарного файла базы данных, содержащий сигнатуру (4 байта), номер транзакции, количество записей и CRC-32 контрольную сумму.

    \item[6.] Разработана полноценная программа управления бинарной базой данных с использованием библиотеки \texttt{ncurses} для создания псевдографического интерфейса. Программа поддерживает:
    \begin{itemize}
        \item создание базы данных при её отсутствии с вводом массива записей структуры \texttt{ZNAK},
        \item поиск людей по году рождения и знаку зодиака с выводом результатов,
        \item добавление новых записей в базу данных с автоматическим обновлением номера транзакции и CRC-32.
    \end{itemize}
    Благодаря использованию \texttt{ncurses}, интерфейс программы адаптируется к размеру терминала и обеспечивает постраничный вывод данных.

    \item[7.] С помощью утилиты \texttt{hexdump} были зафиксированы промежуточные состояния бинарного файла базы после различных операций (создание, добавление записей), что подтвердило изменение контрольной суммы и номера транзакции в заголовке при каждом обновлении базы.

    \item[8.] Для всех программ и библиотек использовались стандартные средства GNU/Linux: компилятор \texttt{gcc}, система сборки \texttt{cmake}, пакет \break\texttt{build-essential}, а также установлены и использованы библиотеки \texttt{ncurses} и \texttt{zlib}, необходимые для работы с текстовым интерфейсом и расчёта CRC-32 соответственно.
\end{itemize}

В итоге были освоены практические приёмы работы со статическими и динамическими библиотеками, управление зависимостями, создание интерфейсов в текстовом режиме с помощью \texttt{ncurses}, работа с бинарными файлами и их контроль целостности с использованием \texttt{zlib}. Кроме того, получены навыки работы в терминальной среде Linux и использования инструментов для анализа зависимостей и отладки бинарных файлов.
