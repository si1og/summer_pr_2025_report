\section*{Введение}
\addcontentsline{toc}{section}{Введение}

Проводилось изучение и применение технологий виртуализации, взаимодействия с операционной системой семейства GNU/Linux, а также разработка и тестирование программного обеспечения на языке программирования Си. Основной целью было закрепление навыков работы в текстовом терминале без графической оболочки, настройка изолированной среды разработки, изучение механизма линковки и работы с библиотеками, включая исследование зависимости исполняемых файлов от разделяемых библиотек.

В качестве среды виртуализации использовалась VirtualBox, установленная на машине с архитектурой ARM64 (Apple Silicon, процессор Apple M3), что потребовало выбора соответствующего дистрибутива Debian 12 под архитектуру \texttt{arm64}. Гостевая система устанавливалась в режиме без графической оболочки, что обеспечило минимальный размер системы.

В процессе выполнения заданий были задействованы следующие технологии: удалённый доступ по протоколу SSH, асимметричная криптография для настройки беспарольной аутентификации, работа с компилятором \texttt{gcc}, построение статических и динамических библиотек, сборка проектов с помощью \texttt{cmake}, анализ зависимостей с помощью \texttt{ldd}, а также создание интерфейсов в псевдографике с использованием библиотеки \texttt{ncurses}.

В последующих разделах отчёта будут представлены результаты настройки виртуальной среды, реализации и тестирования программ, а также анализа их производительности. Будут рассмотрены особенности работы со статическими и динамическими библиотеками, влияние различных флагов оптимизации компилятора на скорость выполнения программ, а также процесс разработки и отладки программ, взаимодействующих с бинарными файлами и пользовательским вводом в текстовом интерфейсе.
