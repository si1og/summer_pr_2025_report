\subsection{Основы работы в командном интерпретаторе}

\subsubsection{Подключение к командному интерпретатору через SSH и проверка службы}

Для удалённого доступа к гостевой операционной системе Debian 12 с хостовой машины (macOS) использовался встроенный SSH-клиент. Подключение производилось к адресу \texttt{127.0.0.1} и проброшенному порту \texttt{2222}, настроенному ранее в VirtualBox. Команда подключения выглядела следующим образом:

\begin{verbatim}
ssh semenov@127.0.0.1 -p 2222
\end{verbatim}

При первом подключении была выведена информация о ключе хоста и предупреждение об отсутствии записи в \texttt{known\_hosts}. После подтверждения и успешного ввода пароля пользователь получил доступ к командному интерпретатору гостевой системы. На скриншоте \ref{fig:Screenshot_2025-06-18_at_20.04.45.png} зафиксирован момент установления соединения.

\screenshot{Screenshot_2025-06-18_at_20.04.45.png}{Успешное подключение к командному интерпретатору через SSH с хостовой системы.}

Для проверки состояния SSH-сервера внутри гостевой ОС была выполнена команда:

\begin{verbatim}
systemctl status ssh
\end{verbatim}

Результат подтвердил, что служба \texttt{ssh.service} активна и запущена. Это позволяет использовать как вход с паролем, так и последующую настройку беспарольной аутентификации. Также на скриншоте \ref{fig:VirtualBox_summer_p_semenov_18_06_2025_20_02_29.png} видно отсутствие утилиты \texttt{sudo}, что объясняется минимальной установкой системы без дополнительных пакетов.

\screenshot{VirtualBox_summer_p_semenov_18_06_2025_20_02_29.png}{Проверка статуса службы OpenSSH на стороне гостевой ОС Debian 12.}

\subsubsection{Редактирование команд в командной строке}

Одной из ключевых особенностей командной оболочки Linux является возможность быстрого и удобного редактирования вводимых команд. Это позволяет эффективно исправлять ошибки, перемещаться по строке и редактировать команды без необходимости повторного ввода.

\begin{itemize}
    \item \textbf{←}, \textbf{→}~--- перемещение курсора влево и вправо по символам.
    \item \textbf{BackSpace}~--- удаление символа слева от курсора.
    \item \textbf{Del}~--- удаление символа под курсором (справа).
    \item \textbf{\^{ }h}~--- эквивалентно \texttt{BackSpace}; используется как клавиатурное сочетание \texttt{Ctrl + H}.
    \item \textbf{\^{ }w}~--- удаляет слово слева от курсора.
    \item \textbf{\^{ }u}~--- удаляет всю строку от начала до текущего положения курсора.
\end{itemize}

Данные клавиши особенно полезны при работе в терминале без графического интерфейса, когда перепечатывание длинных команд может привести к ошибкам и потере времени. Также это хорошая альтернатива мыши при удалённой работе через SSH.

\vspace{1em}
\noindent\textbf{Пример:}

Ввод команды с ошибкой: \texttt{touch /hoem/user/test.txt}
В имени директории опечатка: \texttt{/hoem} вместо \texttt{/home}.
Курсор перемещается с помощью клавиш \texttt{←} до начала слова \texttt{hoem}, затем нажимается \texttt{Ctrl + w} для удаления неверного слова. Вводится корректный путь: \texttt{home}, в итоге команда становится \texttt{touch /home/user/test.txt}.

\subsubsection*{Автодополнение и прерывание команд}

Одной из удобных возможностей командного интерпретатора является автодополнение команд с помощью клавиши \texttt{Tab}. Эта функция позволяет не только ускорить ввод, но и избежать опечаток в командах. В рамках практики было протестировано автодополнение на примере нескольких базовых утилит.

При вводе первых символов команды, например \texttt{up}, нажатие клавиши \texttt{Tab} выводит доступные совпадения: \texttt{update-alternatives}, \texttt{uptime} и др. Далее пользователь может либо продолжить ввод вручную, либо нажать \texttt{Tab} повторно для автозавершения, если совпадение однозначно.

На скриншоте \ref{fig:Screenshot_2025-06-26_at_09.47.08.png} приведён пример автодополнения команды \texttt{uptime} и вызова команды \texttt{time} без аргументов, что выводит отчёт с нулевыми значениями времени.

\screenshot{Screenshot_2025-06-26_at_09.47.08.png}{Автодополнение команд и использование утилиты \texttt{time}.}

Также в процессе практики были протестированы механизмы прерывания команд:

\begin{itemize}
    \item \texttt{Ctrl + C} --- немедленное прерывание текущей команды или процесса.
    \item \texttt{q} --- завершение вывода в утилитах постраничного просмотра (\texttt{less}, \texttt{man}, \texttt{hexdump}).
    \item \texttt{Ctrl + D} --- отправка признака конца ввода (EOF), используется для выхода из ввода или завершения сессии.
\end{itemize}

\subsubsection*{Использование справочной системы \texttt{man}}

В операционной системе GNU/Linux для получения справки по командам и системным вызовам используется утилита \texttt{man}. Справочные страницы (\textit{man pages}) содержат описание синтаксиса, аргументов, ключей и примеры использования.

При вызове справки по команде \texttt{cp} с помощью:

\begin{verbatim}
man cp
\end{verbatim}

\noindentбыло выведено предупреждение (рис. \ref{fig:Screenshot_2025-06-26_at_09.59.25.png}):
\texttt{man: can't set the locale; make sure \$LC\_* and \$LANG are correct}.
Оно связано с тем, что в минимальной установке Debian отсутствуют настройки локали. Тем не менее, справочная страница отобразилась корректно (рис. \ref{fig:Screenshot_2025-06-26_at_09.55.49.png}).

\screenshot{Screenshot_2025-06-26_at_09.55.49.png}{Справка по команде \texttt{cp}.}

\texttt{man}-страница включает описание команды \texttt{cp} и возможных ключей (например, \texttt{-r}, \texttt{-a}, \texttt{--backup}, и др.).

\screenshot{Screenshot_2025-06-26_at_09.59.25.png}{Ошибка локали при запуске \texttt{man cp}.}

Полученные сведения были сразу же применены при выполнении команды копирования. В ходе практики был создан файл \texttt{test.txt}, который затем копировался в подкаталог \texttt{politech}. Команда:

\begin{verbatim}
cp ./* politech
\end{verbatim}

\noindentотработала только для файлов, но не для подкаталогов, что привело к сообщению:

\begin{verbatim}
cp: -r not specified; omitting directory './politech'
\end{verbatim}

\noindentПосле выполнения команды \texttt{ls politech} можно убедиться, что файл \texttt{test.txt} был успешно скопирован, а каталоги были проигнорированы из-за отсутствия ключа \texttt{-r} (рис. \ref{fig:Screenshot_2025-06-26_at_11.47.45.png}).

\screenshot{Screenshot_2025-06-26_at_11.47.45.png}
{Практическое применение информации из \texttt{man cp}.}

\subsubsection{Назначение команд и целевые действия}

В ходе практики были опробованы и изучены базовые команды GNU/Linux, перечисленные разделе \textbf{\flqqОсновы работы в командном интерпретаторе\frqq}. Для каждой команды была открыта справочная страница с помощью \texttt{man} и выполнены практические действия. Ниже приведён краткий обзор:

\begin{itemize}
    \item \texttt{whoami}~--- выводит имя текущего пользователя.
    
    \item \texttt{date}~--- отображает текущие дату и время.

    \item \texttt{cal}~--- показывает календарь на текущий или заданный месяц/год.

    \item \texttt{uptime}~--- показывает время работы системы, количество пользователей и среднюю загрузку.

    \item \texttt{df}~--- показывает информацию об использовании дискового пространства (по умолчанию в 1К-блоках).

    \item \texttt{cp}~--- копирует файлы и каталоги. Пример описан в разделе выше.

    \item \texttt{mv}~--- перемещает (или переименовывает) файлы и каталоги.

    \item \texttt{touch}~--- создаёт новый пустой файл или обновляет временные метки существующего.

    \item \texttt{rm}~--- удаляет файлы и каталоги. Требует осторожности, особенно с опцией \texttt{-r}.

    \item \texttt{stat}~--- отображает подробную информацию о файле: размер, время доступа и изменения, права.

    \item \texttt{file}~--- определяет тип содержимого файла.

    \item \texttt{find}~--- ищет файлы и каталоги по имени и другим критериям.

    \item \texttt{whereis}~--- показывает расположение бинарного файла, исходников и справки по имени команды.

    \item \texttt{cat}~--- выводит содержимое файла в стандартный вывод.

    \item \texttt{hexdump}~--- отображает бинарное содержимое файла в шестнадцатеричном виде.

    \item \texttt{less}~--- постраничный просмотр длинных текстов, например:

    \item \texttt{grep}~--- поиск по шаблону (регулярному выражению) в текстовых данных.

    \item \texttt{time}~--- измеряет продолжительность выполнения команды. Без аргументов просто выводит нули.

\end{itemize}

\subsubsection{Примеры выполнения команд и их результатов}

На скриншоте \ref{fig:Screenshot_2025-06-26_at_12.28.01} показан вывод ряда базовых команд: \texttt{whoami}, \texttt{date}, \texttt{uptime} и \texttt{df}. Команда \texttt{whoami} отобразила имя текущего пользователя, \texttt{date} вывела текущие дату и время, \texttt{uptime} сообщила о времени непрерывной работы системы и нагрузке на процессор, а \texttt{df} показала информацию об использовании дискового пространства. Попытка вызова команды \texttt{cal} завершилась ошибкой, так как утилита не была установлена в системе.
\screenshot{Screenshot_2025-06-26_at_12.28.01}{Вывод базовых команд.}

Затем был создан каталог \texttt{files} с помощью команды \texttt{mkdir}, а также пустой файл \texttt{file.txt} командой \texttt{touch}. Это подготовило рабочую структуру для выполнения последующих операций с файлами (рис. \ref{fig:Screenshot_2025-06-26_at_12.28.52}).
\screenshot{Screenshot_2025-06-26_at_12.28.52}{Создание каталога и файла.}

Содержимое файла \texttt{file.txt} было отредактировано в терминальном текстовом редакторе \texttt{nano}. В файл была добавлена строка \texttt{Hello!}, после чего он был сохранён и закрыт (рис. \ref{fig:Screenshot_2025-06-26_at_12.29.30}).
\screenshot{Screenshot_2025-06-26_at_12.29.30}{Редактирование файла в \texttt{nano}.}

На следующем этапе показано использование команд \texttt{mv}, \texttt{cp} и \texttt{rm}. Файл был перемещён в подкаталог, затем скопирован обратно в текущую директорию, а после этого удалён. Также была применена команда \texttt{stat}, позволившая получить расширенную информацию о файле: его размер, права доступа, дату создания и последнего изменения (рис. \ref{fig:Screenshot_2025-06-26_at_12.31.35}).
\screenshot{Screenshot_2025-06-26_at_12.31.35}{Работа с файлами и метаданными.}

Команда \texttt{file} определила тип содержимого файла как текстовый. Утилита \texttt{find} успешно нашла все \texttt{.txt}-файлы в текущем каталоге. С помощью \texttt{whereis} был определён путь к исполняемому файлу \texttt{cp} и его документации. Затем выполнены команды \texttt{cat} и \texttt{hexdump}, продемонстрировавшие содержимое файла в обычном и шестнадцатеричном представлении соответственно (рис. \ref{fig:Screenshot_2025-06-26_at_12.34.36}).
\screenshot{Screenshot_2025-06-26_at_12.34.36}{Анализ и просмотр файлов.}

Далее файл был открыт с помощью \texttt{less} — удобного инструмента для просмотра содержимого с возможностью прокрутки. Управление просмотром осуществляется клавишами, а выход из режима просмотра выполняется нажатием \texttt{q} (рис. \ref{fig:Screenshot_2025-06-26_at_12.34.47}).
\screenshot{Screenshot_2025-06-26_at_12.34.47}{Просмотр текста через \texttt{less}.}

Наконец, команда \texttt{time} была вызвана без аргументов, что привело к мгновенному завершению с нулевыми значениями времени исполнения. Также показано удаление каталога \texttt{files} с помощью \texttt{rm -rf} (рис. \ref{fig:Screenshot_2025-06-26_at_12.35.35}).
\screenshot{Screenshot_2025-06-26_at_12.35.35}{Удаление каталога и использование \texttt{time}.}

\subsubsection{Работа с историей команд}

В командной строке GNU/Linux доступна удобная навигация по ранее введённым командам с помощью клавиш со стрелками: \textbf{↑}~— для перехода к предыдущим командам, \textbf{↓}~— к последующим. Это позволяет быстро повторять, корректировать или комбинировать уже выполненные команды без необходимости повторного ручного ввода.

Для вывода полной истории всех команд используется команда \texttt{history}. На скриншоте \ref{fig:Screenshot_2025-06-26_at_14.18.22} показан результат её выполнения.

Последними тремя командами в истории стали:
\begin{enumerate}
    \item \texttt{man cal}~— попытка обратиться к справке по команде \texttt{cal};
    \item \texttt{ls}~— вывод содержимого текущего каталога;
    \item \texttt{rm -rf files}~— удаление каталога \texttt{files} со всем его содержимым.
\end{enumerate}

\screenshot{Screenshot_2025-06-26_at_14.18.22}{История команд, полученная с помощью \texttt{history}.}

\subsubsection*{Настройка аутентификации по ключу}

Важной частью работы с терминалом и удалёнными серверами является настройка безопасного и удобного способа входа — с помощью пары приватного и публичного ключей. Это исключает необходимость каждый раз вводить пароль, а также повышает безопасность подключения.

На первом этапе была сгенерирована новая пара ключей с помощью команды:
\begin{verbatim}
ssh-keygen -t ed25519 -C "practice_key"
\end{verbatim}
При этом была создана пара файлов \verb|id_ed25519| (приватный ключ) и \verb|id_ed25519.pub| (публичный ключ) в домашнем каталоге пользователя macOS. Это показано на скриншоте \ref{fig:Screenshot_2025-06-19_at_17.18.44.png}.
\screenshot{Screenshot_2025-06-19_at_17.18.44.png}{Генерация SSH-ключа.}.

Затем публичный ключ был передан в виртуальную машину с Debian 12 при помощи команды:
\begin{verbatim}
ssh-copy-id -p 2222 semenov@127.0.0.1
\end{verbatim}
После ввода пароля ключ был успешно добавлен в \verb|~/.ssh/authorized_keys| на гостевой системе. Скриншот \ref{fig:Screenshot_2025-06-19_at_17.19.38.png} демонстрирует этот процесс.

\screenshot{Screenshot_2025-06-19_at_17.19.38.png}{Передача ключа с помощью ssh-copy-id.}

В завершение была предпринята попытка подключиться к гостевой машине без ввода пароля:
\begin{verbatim}
ssh -p 2222 semenov@127.0.0.1
\end{verbatim}
Как видно на скриншоте \ref{fig:Screenshot_2025-06-19_at_17.20.29.png}, подключение прошло успешно, а пароль больше не требуется.

\screenshot{Screenshot_2025-06-19_at_17.20.29.png}{Успешное подключение по ключу.}

Таким образом, аутентификация по SSH-ключу была успешно настроена, что позволяет работать с виртуальной машиной быстрее и безопаснее.
