\subsection{Разработка программного обеспечения}

\subsubsection*{Установка средств разработки}

В рамках раздела \textbf{«Разработка программного обеспечения»} была выполнена установка базового комплекта инструментов для сборки C/C++ программ.

При попытке установить пакет \texttt{build-essential} от имени обычного пользователя возникла ошибка (см. скриншот \ref{fig:Screenshot_2025-06-19_at_17.22.16.png}). 

\screenshot{Screenshot_2025-06-19_at_17.22.16.png}{Пользователь не имеет прав на установку пакетов и команда \texttt{sudo} не найдена}

Для решения проблемы была произведена авторизация под суперпользователем с помощью команды \texttt{su}, после чего были обновлены списки пакетов (\texttt{apt update}) и установлен пакет \texttt{sudo} (см. скриншот \ref{fig:VirtualBox_summer_p_semenov_19_06_2025_17_49_52.png}). 

\screenshot{VirtualBox_summer_p_semenov_19_06_2025_17_49_52.png}{Демонстрирует установку \texttt{sudo}}

Затем, чтобы предоставить пользователю \texttt{semenov} возможность использовать \texttt{sudo} без необходимости входа под \texttt{root}, он был добавлен в группу \texttt{sudo} с помощью команды:

\begin{verbatim}
usermod -aG sudo semenov
\end{verbatim}

После выхода из текущего сеанса и повторного входа пользователь получил возможность выполнять команды с повышенными правами, используя \texttt{sudo}.

Далее была успешно выполнена команда \texttt{sudo apt install build-essential}, как показано на скриншоте \ref{fig:Screenshot_2025-06-19_at_18.01.20.png}, включающая в себя компиляторы \texttt{gcc}, \texttt{g++}, отладочные утилиты, заголовочные файлы и библиотеки.

\screenshot{Screenshot_2025-06-19_at_18.01.20.png}{Установка средств разработки}

\subsubsection*{Анализ зависимостей \texttt{/bin/bash}}

При помощи команды \texttt{ldd /bin/bash} было исследовано, какие динамические библиотеки используются командным интерпретатором. Затем аналогичная проверка была выполнена для каждой зависимости рекурсивно, включая \texttt{libtinfo.so.6}, \texttt{libc.so.6} и \texttt{ld-linux-aarch64.so.1}. Полученное дерево зависимостей:

\usetikzlibrary{trees}

\begin{figure}[H]
\centering
\begin{tikzpicture}[
  grow=down,
  level distance=1.8cm,
  sibling distance=5cm,
  edge from parent/.style={draw, -latex},
  every node/.style={draw=black, rounded corners, font=\ttfamily, align=left}
  ]

\node {/bin/bash}
  child { node {libtinfo.so.6}
    child { node {libc.so.6}
      child { node {ld-linux-aarch64.so.1} }
    }
  }
  child { node {libc.so.6}
    child { node {ld-linux-aarch64.so.1} }
  }
  child { node {ld-linux-aarch64.so.1\\(статически связано)} };

\end{tikzpicture}
\caption{Рекурсивное дерево зависимостей \texttt{/bin/bash}, построенное с использованием \texttt{ldd}}
\label{fig:bash-ldd-tree}
\end{figure}

Для построения дерева использовалась утилита \texttt{ldd}, а также информация, полученная вручную при проверке каждой библиотеки по отдельности (см. скриншот \ref{fig:Screenshot_2025-06-26_at_16.20.17.png}).

\screenshot{Screenshot_2025-06-26_at_16.20.17.png}{рекурсивное исследование зависимостей с помощью \texttt{ldd}}

\subsubsection*{Анализ других программ}

С помощью команды \texttt{ldd} было также проанализировано количество зависимостей у основных утилит из п.~3 раздела <<Основы работы в командном интерпретаторе>>. Ниже приведено краткое резюме:

\begin{itemize}
    \item \texttt{whoami}, \texttt{date}, \texttt{cal}, \texttt{df}, \texttt{cat}, \texttt{touch}, \texttt{rm}, \texttt{whereis}~--- имеют минимальный набор зависимостей: как правило, это \texttt{libc.so.6}, загрузчик \break\texttt{ld-linux-aarch64.so.1}, а также виртуальная библиотека \break\texttt{linux-vdso.so.1}.

    \item \texttt{hexdump}, \texttt{less}~--- дополнительно используют библиотеку \texttt{libtinfo.so.6}, обеспечивающую взаимодействие с терминалом (цвет, прокрутка и т.д.).

    \item \texttt{stat}, \texttt{find}, \texttt{mv}, \texttt{cp}~--- используют расширенный набор библиотек: помимо \texttt{libc}, задействованы \texttt{libselinux.so.1}, \texttt{libacl.so.1}, \texttt{libattr.so.1}, а также регулярные выражения (\texttt{libpcre2-8.so.0}).

    \item \texttt{file}~— одна из самых зависимых утилит, использующая до 7 динамических библиотек, включая \texttt{libmagic}, \texttt{libz}, \texttt{liblzma}, \texttt{libbz2}, что связано с определением форматов и сжатыми файлами.

    \item \texttt{grep}~— использует \texttt{libpcre2-8.so.0} для поиска по шаблонам.

    \item \texttt{uptime}~— задействует широкую цепочку библиотек: помимо \texttt{libproc2}, используются \texttt{libsystemd}, \texttt{libcap}, \texttt{liblzma}, \texttt{libzstd}, \texttt{liblz4}, \texttt{libgpg-error}, \texttt{libgcrypt} и др.

    \item \texttt{time}~— отсутствует как отдельная бинарная утилита в системе, возможно реализована как shell built-in или отсутствует в составе базового окружения (скриншот \ref{fig:Screenshot_2025-06-26_at_17.06.14.png}).
\end{itemize}

\screenshot{Screenshot_2025-06-26_at_17.06.14.png}{Вызов \texttt{ldd} для \texttt{time}}

Количество и состав библиотек может варьироваться в зависимости от сборки и архитектуры системы (в данном случае — \texttt{aarch64}), однако во всех случаях ядром зависимостей остаётся стандартная библиотека \texttt{libc.so.6} и динамический загрузчик \texttt{ld-linux-aarch64.so.1}.

В качестве подтверждения приведены скриншоты \ref{fig:Screenshot_2025-06-26_at_16.47.02.png}, \ref{fig:Screenshot_2025-06-26_at_16.47.37.png}, \ref{fig:Screenshot_2025-06-26_at_16.48.18.png}, \ref{fig:Screenshot_2025-06-26_at_16.49.59.png} и \ref{fig:Screenshot_2025-06-26_at_16.51.37.png} с выводом команды \texttt{ldd} для соответствующих утилит:

\screenshot{Screenshot_2025-06-26_at_16.47.02.png}{ldd для \texttt{hexdump}, \texttt{find}, \texttt{whereis}}

\screenshot{Screenshot_2025-06-26_at_16.47.37.png}{ldd для \texttt{cat}, \texttt{less}, \texttt{grep}}

\screenshot{Screenshot_2025-06-26_at_16.48.18.png}{ldd для \texttt{touch}, \texttt{rm}, \texttt{stat}, \texttt{file}}

\screenshot{Screenshot_2025-06-26_at_16.49.59.png}{ldd для \texttt{df}, \texttt{cp}, \texttt{mv}}

\screenshot{Screenshot_2025-06-26_at_16.51.37.png}{ldd для \texttt{whoami}, \texttt{date}, \texttt{uptime}}

\subsubsection*{Установка  \texttt{CMake}, отличие системы сборки и мета-системы}

\textbf{Система сборки} — это инструмент, который напрямую управляет компиляцией и сборкой проекта. Одним из классических примеров является \texttt{make}, использующий файл \texttt{Makefile} с набором правил. Пример простейшего \texttt{Makefile}:

\begin{verbatim}
all:
	$(CC) main.c -o main
\end{verbatim}

\textbf{Мета-система сборки}, такая как \texttt{CMake}, не выполняет сборку сама по себе. Она служит для генерации файлов сборки, адаптированных под конкретную систему или генератор (например, Makefile). Это особенно полезно при переносе проекта на другие платформы и конфигурации.

Для установки \texttt{cmake} была использована команда \texttt{sudo apt install cmake}.

\screenshot{Screenshot_2025-06-20_at_00.42.27.png}{Установка \texttt{cmake} и зависимостей через \texttt{apt}}

\subsubsection*{Структура файла \texttt{CMakeLists.txt}}

Для каждого задания был создан отдельный файл \texttt{CMakeLists.txt}~--- далее это будет показато в отчёте, --- содержащий инструкции по сборке. Ниже приведён пример структуры этого файла для проекта, включающего исходный файл и библиотеку:

\begin{lstlisting}[language=bash,caption={Пример структуры CMakeLists.txt}]
cmake_minimum_required(VERSION 3.10)

# Название проекта
project(MaxLibrary C)

# Добавление флагов компиляции (по желанию)
set(CMAKE_C_STANDARD 99)

# Указание исходных файлов
add_library(maxlib STATIC max.c)
# или для динамической библиотеки:
# add_library(maxlib SHARED max.c)

# Добавление исполняемого файла, использующего библиотеку
add_executable(main main.c)

# Линковка исполняемого файла с библиотекой
target_link_libraries(main maxlib)
\end{lstlisting}

\begin{itemize}
  \item \texttt{cmake\_minimum\_required}~--- указывает минимально необходимую версию CMake.
  \item \texttt{project}~--- задаёт имя проекта и язык программирования.
  \item \texttt{add\_library}~--- создаёт статическую или динамическую библиотеку из исходного файла.
  \item \texttt{add\_executable}~--- создаёт исполняемый файл.
  \item \texttt{target\_link\_libraries}~--- связывает исполняемый файл с библиотекой.
\end{itemize}