\subsection{Разработка программного обеспечения}

\subsubsection{Установка средств разработки}

В рамках раздела \textbf{«Разработка программного обеспечения»} была выполнена установка базового комплекта инструментов для сборки C/C++ программ.

При попытке установить пакет \texttt{build-essential} от имени обычного пользователя возникла ошибка (рис. \ref{fig:Screenshot_2025-06-19_at_17.22.16.png}). 

\screenshot{Screenshot_2025-06-19_at_17.22.16.png}{Пользователь не имеет прав на установку пакетов и команда \texttt{sudo} не найдена}

Для решения проблемы была произведена авторизация под суперпользователем с помощью команды \texttt{su}, после чего были обновлены списки пакетов (\texttt{apt update}) и установлен пакет \texttt{sudo} (рис. \ref{fig:VirtualBox_summer_p_semenov_19_06_2025_17_49_52.png}). 

\screenshot{VirtualBox_summer_p_semenov_19_06_2025_17_49_52.png}{Демонстрирует установку \texttt{sudo}.}

Затем, чтобы предоставить пользователю \texttt{semenov} возможность использовать \texttt{sudo} без необходимости входа под \texttt{root}, он был добавлен в группу \texttt{sudo} с помощью команды:

\begin{verbatim}
usermod -aG sudo semenov
\end{verbatim}

После выхода из текущего сеанса и повторного входа пользователь получил возможность выполнять команды с повышенными правами, используя \texttt{sudo}.

Далее была успешно выполнена команда \texttt{sudo apt install build-essential}, как показано на скриншоте \ref{fig:Screenshot_2025-06-19_at_18.01.20.png}, включающая в себя компиляторы \texttt{gcc}, \texttt{g++}, отладочные утилиты, заголовочные файлы и библиотеки.

\screenshot{Screenshot_2025-06-19_at_18.01.20.png}{Установка средств разработки.}

\subsubsection*{Состав пакета \texttt{build-essential}}

Метапакет \texttt{build-essential} включает следующий минимальный набор пакетов, необходимых для сборки программ из исходных кодов на C и C++, а также для сборки Debian-пакетов:

\begin{itemize}
    \item \texttt{gcc} --- компилятор языка C (GNU Compiler Collection).
    \item \texttt{g++} --- компилятор языка C++ (GNU Compiler Collection).
    \item \texttt{make} --- инструмент для автоматизации сборки проектов на основе Makefile.
    \item \texttt{libc6-dev} --- заголовочные файлы стандартной библиотеки C (glibc), необходимые для компиляции программ под Linux.
    \item \texttt{libc-dev} --- псевдоним для \texttt{libc6-dev}, поддерживаемый для совместимости.
    \item \texttt{dpkg-dev} --- пакет, содержащий набор инструментов для создания и сопровождения исходных пакетов Debian. В него входят:
    \begin{itemize}
        \item \texttt{dpkg-source} --- программа для распаковки и сборки исходных пакетов Debian, управления патчами.
        \item \texttt{dpkg-gencontrol} --- генерирует файл \texttt{DEBIAN/control}, содержащий метаданные бинарного пакета на основе информации из исходного пакета.
        \item \texttt{dpkg-shlibdeps} --- анализирует бинарные файлы на предмет использования разделяемых библиотек и автоматически вычисляет зависимости \texttt{Depends}.
        \item \texttt{dpkg-genchanges} --- формирует файл \texttt{.changes}, содержащий сведения о построенных пакетах и их проверках, который нужен для загрузки в репозиторий.
        \item \texttt{dpkg-buildpackage} --- основной скрипт для автоматизированной сборки бинарных и исходных пакетов Debian из каталога с исходниками.
        \item \texttt{dpkg-distaddfile} --- добавляет информацию о дополнительных файлах в файл \texttt{debian/files}, используемый при сборке.
    \end{itemize}
\end{itemize}

\subsubsection*{Использование \texttt{sudo} и \texttt{apt} для управления пакетами}

В операционных системах семейства GNU/Linux, таких как Debian, управление программным обеспечением чаще всего осуществляется через менеджер пакетов \texttt{APT} (Advanced Package Tool). Он позволяет устанавливать, обновлять и удалять пакеты программ вместе с их зависимостями из репозиториев.

Для выполнения подобных действий обычно требуются привилегии суперпользователя (\texttt{root}). Для того чтобы не переключаться в сессию \texttt{root}, используется программа \texttt{sudo}, которая позволяет запускать отдельные команды с правами администратора от имени обычного пользователя, входящего в группу \texttt{sudo}.

Например, команды для установки и обновления пакетов могли выглядеть следующим образом:
\begin{verbatim}
sudo apt update
sudo apt install build-essential
\end{verbatim}
где:
\begin{itemize}
    \item \texttt{sudo} --- позволяет запустить команду от имени администратора.
    \item \texttt{apt} --- консольная утилита для работы с пакетами, предоставляющая удобный интерфейс для установки, удаления и обновления пакетов.
    \item \texttt{update} --- обновляет локальный кеш списка пакетов из всех подключенных репозиториев.
    \item \texttt{install <пакет>} --- устанавливает указанный пакет и все необходимые зависимости.
\end{itemize}

Кроме того, важную роль в управлении пакетами играют зеркала репозиториев — специальные серверы, хранящие каталоги пакетов для выбранного дистрибутива. При установке Debian в VirtualBox в качестве зеркала был выбран официальный сервер \texttt{deb.debian.org}, который автоматически перенаправляет запросы пользователя на ближайший и наиболее доступный зеркальный сервер, что повышает скорость загрузки и устойчивость к сбоям отдельных узлов. При запуске команды \texttt{sudo apt update} утилита \texttt{apt} обращается к списку зеркал в файле \texttt{/etc/apt/sources.list}, загружает информацию о доступных версиях пакетов и обновляет локальный кэш метаданных, позволяя затем быстро устанавливать или обновлять программное обеспечение из проверенных источников.

\subsubsection{Анализ зависимостей \texttt{/bin/bash}}

При помощи команды \texttt{ldd /bin/bash} было исследовано, какие динамические библиотеки используются командным интерпретатором. Затем аналогичная проверка была выполнена для каждой зависимости рекурсивно, включая \texttt{libtinfo.so.6}, \texttt{libc.so.6} и \texttt{ld-linux-aarch64.so.1}. Полученное дерево зависимостей:

\usetikzlibrary{trees}

\begin{figure}[H]
\centering
\begin{tikzpicture}[
  grow=down,
  level distance=1.8cm,
  sibling distance=5cm,
  edge from parent/.style={draw, -latex},
  every node/.style={draw=black, rounded corners, font=\ttfamily, align=left}
  ]

\node {/bin/bash}
  child { node {libtinfo.so.6}
    child { node {libc.so.6}
      child { node {ld-linux-aarch64.so.1} }
    }
  }
  child { node {libc.so.6}
    child { node {ld-linux-aarch64.so.1} }
  }
  child { node {ld-linux-aarch64.so.1\\(статически связано)} };

\end{tikzpicture}
\caption{Рекурсивное дерево зависимостей \texttt{/bin/bash}, построенное с использованием \texttt{ldd}}
\label{fig:bash-ldd-tree}
\end{figure}

Для построения дерева использовалась утилита \texttt{ldd}, а также информация, полученная вручную при проверке каждой библиотеки по отдельности (рис. \ref{fig:Screenshot_2025-06-26_at_16.20.17.png}).

\screenshot{Screenshot_2025-06-26_at_16.20.17.png}{рекурсивное исследование зависимостей с помощью \texttt{ldd}.}

\subsubsection{Анализ зависимостей других программ}

С помощью команды \texttt{ldd} было также проанализировано количество зависимостей у основных утилит из п.~3 раздела <<Основы работы в командном интерпретаторе>>. Ниже приведено краткое резюме:

\begin{itemize}
    \item \texttt{whoami}, \texttt{date}, \texttt{cal}, \texttt{df}, \texttt{cat}, \texttt{touch}, \texttt{rm}, \texttt{whereis}~--- имеют минимальный набор зависимостей: как правило, это \texttt{libc.so.6}, загрузчик \break\texttt{ld-linux-aarch64.so.1}, а также виртуальная библиотека \break\texttt{linux-vdso.so.1}.

    \item \texttt{hexdump}, \texttt{less}~--- дополнительно используют библиотеку \texttt{libtinfo.so.6}, обеспечивающую взаимодействие с терминалом (цвет, прокрутка и т.д.).

    \item \texttt{stat}, \texttt{find}, \texttt{mv}, \texttt{cp}~--- используют расширенный набор библиотек: помимо \texttt{libc}, задействованы \texttt{libselinux.so.1}, \texttt{libacl.so.1}, \texttt{libattr.so.1}, а также регулярные выражения (\texttt{libpcre2-8.so.0}).

    \item \texttt{file}~— одна из самых зависимых утилит, использующая до 7 динамических библиотек, включая \texttt{libmagic}, \texttt{libz}, \texttt{liblzma}, \texttt{libbz2}, что связано с определением форматов и сжатыми файлами.

    \item \texttt{grep}~— использует \texttt{libpcre2-8.so.0} для поиска по шаблонам.

    \item \texttt{uptime}~— задействует широкую цепочку библиотек: помимо \texttt{libproc2}, используются \texttt{libsystemd}, \texttt{libcap}, \texttt{liblzma}, \texttt{libzstd}, \texttt{liblz4}, \texttt{libgpg-error}, \texttt{libgcrypt} и др.

    \item \texttt{time}~— отсутствует как отдельная бинарная утилита в системе, возможно реализована как shell built-in или отсутствует в составе базового окружения (рис. \ref{fig:Screenshot_2025-06-26_at_17.06.14.png}).
\end{itemize}

\screenshot{Screenshot_2025-06-26_at_17.06.14.png}{Вызов \texttt{ldd} для \texttt{time}.}

Количество и состав библиотек может варьироваться в зависимости от сборки и архитектуры системы (в данном случае — \texttt{aarch64}), однако во всех случаях ядром зависимостей остаётся стандартная библиотека \texttt{libc.so.6} и динамический загрузчик \texttt{ld-linux-aarch64.so.1}.

В качестве подтверждения приведены скриншоты \ref{fig:Screenshot_2025-06-26_at_16.47.02.png}, \ref{fig:Screenshot_2025-06-26_at_16.47.37.png}, \ref{fig:Screenshot_2025-06-26_at_16.48.18.png}, \ref{fig:Screenshot_2025-06-26_at_16.49.59.png} и \ref{fig:Screenshot_2025-06-26_at_16.51.37.png} с выводом команды \texttt{ldd} для соответствующих утилит:

\screenshot{Screenshot_2025-06-26_at_16.47.02.png}{ldd для \texttt{hexdump}, \texttt{find}, \texttt{whereis}.}

\screenshot{Screenshot_2025-06-26_at_16.47.37.png}{ldd для \texttt{cat}, \texttt{less}, \texttt{grep}.}

\screenshot{Screenshot_2025-06-26_at_16.48.18.png}{ldd для \texttt{touch}, \texttt{rm}, \texttt{stat}, \texttt{file}.}

\screenshot{Screenshot_2025-06-26_at_16.49.59.png}{ldd для \texttt{df}, \texttt{cp}, \texttt{mv}.}

\screenshot{Screenshot_2025-06-26_at_16.51.37.png}{ldd для \texttt{whoami}, \texttt{date}, \texttt{uptime}.}

\subsubsection{Установка  \texttt{CMake}, отличие системы сборки и мета-системы}

\textbf{Система сборки} — это инструмент, который напрямую управляет компиляцией и сборкой проекта. Одним из классических примеров является \texttt{make}, использующий файл \texttt{Makefile} с набором правил. Пример простейшего \texttt{Makefile}:

\begin{verbatim}
all:
	$(CC) main.c -o main
\end{verbatim}

\textbf{Мета-система сборки}, такая как \texttt{CMake}, не выполняет сборку сама по себе. Она служит для генерации файлов сборки, адаптированных под конкретную систему или генератор (например, Makefile). Это особенно полезно при переносе проекта на другие платформы и конфигурации.

Для установки \texttt{cmake} была использована команда \texttt{sudo apt install cmake}. Процесс установки придедён на скриншоте \ref{fig:Screenshot_2025-06-20_at_00.42.27.png}.

\screenshot{Screenshot_2025-06-20_at_00.42.27.png}{Установка \texttt{cmake} и зависимостей через \texttt{apt}.}

\subsubsection*{Описание файла \texttt{CMakeLists.txt}}

Файл \texttt{CMakeLists.txt} используется системой CMake для описания правил сборки проекта. В нём указываются имя проекта, используемый язык программирования, минимальная версия CMake, а также перечисляются исходные файлы, из которых нужно собрать библиотеки и исполняемые программы.

Этот файл позволяет автоматически генерировать Makefile или другие скрипты сборки, чтобы без ручного ввода команд компилировать исходники, собирать статические или динамические библиотеки и связывать их с исполняемыми файлами.

В нашей работе \texttt{CMakeLists.txt} применялся для создания библиотек с функциями поиска максимума, тестовых программ, а также для экспериментов с флагами оптимизации компилятора. Это делало сборку проекта удобной и одинаковой на любой платформе, где установлен CMake.