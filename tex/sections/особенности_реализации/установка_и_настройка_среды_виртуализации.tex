\subsection{Установка и настройка среды виртуализации}

\subsubsection*{Установка среды VirtualBox}

В качестве основной платформы виртуализации использовалась среда \texttt{VirtualBox}~--- свободно распространяемый гипервизор, обеспечивающий изоляцию и запуск гостевых операционных систем. Установка VirtualBox была выполнена на хостовой системе с операционной системой macOS под управлением процессора Apple M3 (архитектура ARM64). Данная среда позволяет создавать виртуальные машины с заданными параметрами, управлять виртуальными сетями и использовать гибкую систему образов дисков. Установка производилась с официального сайта проекта VirtualBox.

\subsubsection*{Установка и загрузка гостевой операционной системы GNU/Linux Debian 12}

Для установки гостевой операционной системы использовался официальный минимальный образ \texttt{debian-12.11.0-arm64-netinst.iso}, скачанный с сайта проекта Debian. Архитектура ARM64 была выбрана в соответствии с платформой Apple M3, на которой запускалась виртуальная машина. Скриншот с сайта загрузки установочного образа приведён ниже:

\screenshot{Screenshot_2025-06-18_at_19.10.15.png}{Страница загрузки установочного образа Debian 12 для ARM64}

Установка производилась в текстовом режиме. На этапе выбора временной зоны были предложены только американские варианты, поскольку ранее в процессе установки в качестве страны по умолчанию были выбраны Соединённые Штаты. При необходимости установки локального времени для России следует вручную выбрать страну «Россия» на предыдущем этапе. В качестве демонстрации представлен экран выбора временной зоны:

\screenshot{VirtualBox_summer_p_semenov_18_06_2025_19_29_24.png}{Выбор часового пояса в процессе установки Debian}

Разметка виртуального диска была выполнена автоматически, с использованием всего пространства на диске. Был выбран метод разметки \texttt{Guided - use entire disk}, как рекомендованный для простых случаев без использования LVM или шифрования. Этот способ минимизирует сложность конфигурации и подходит для учебных задач:

\screenshot{VirtualBox_summer_p_semenov_18_06_2025_19_31_43.png}{Автоматическая разметка виртуального диска при установке Debian}

\subsubsection*{Настройка \flqqпроброса\frqq\ портов для подключения к гостевой ОС по SSH}

Для обеспечения доступа к гостевой системе по протоколу SSH необходимо настроить \flqqпроброс\frqq\ порта 22 из виртуальной машины на свободный порт хостовой операционной системы. Это позволяет подключаться к гостевой Debian из macOS с использованием локального IP-адреса и перенаправленного порта, что особенно удобно при работе через терминал.

В настройках сети виртуальной машины в VirtualBox был использован сетевой адаптер типа \texttt{NAT}. Далее во вкладке \texttt{Port Forwarding} было добавлено новое правило: хостовый порт 2222 (на \texttt{127.0.0.1}) перенаправляется на порт 22 гостевой системы.

Скриншот с настройкой правила \flqqпроброса\frqq\ порта приведён ниже:

\screenshot{Screenshot_2025-06-18_at_19.56.16.png}{Настройка \flqqпроброса\frqq\ 22 порта SSH на 2222 порт хоста в VirtualBox}

\subsubsection*{Приглашение к вводу имени пользователя после загрузки системы}

После завершения установки и перезагрузки гостевой операционной системы Debian 12 система автоматически запускается в текстовом терминале \texttt{tty1}, без графической оболочки. На экране появляется приглашение к вводу имени пользователя, оформленное в виде строки:

\screenshot{VirtualBox_summer_p_semenov_25_06_2025_22_24_04.png}{Экран приглашения к вводу логина после загрузки в терминале \texttt{tty1}}

Это подтверждает успешную загрузку системы, правильную настройку учётной записи и работу без графического интерфейса.

Ниже представлен успешный вход в систему под созданным ранее пользователем \texttt{semenov}:

\screenshot{VirtualBox_summer_p_semenov_18_06_2025_19_57_43.png}{Успешный вход в систему: отображение имени пользователя, имени хоста и приглашения командной строки}
