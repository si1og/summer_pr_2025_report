\subsection{Установка и настройка среды виртуализации}

\subsubsection{Установка среды VirtualBox}

В качестве основной платформы виртуализации использовалась среда \texttt{VirtualBox}~--- свободно распространяемый гипервизор, обеспечивающий изоляцию и запуск гостевых операционных систем. Установка VirtualBox была выполнена на хостовой системе с операционной системой macOS под управлением процессора Apple M3 (архитектура ARM64). Данная среда позволяет создавать виртуальные машины с заданными параметрами, управлять виртуальными сетями и использовать гибкую систему образов дисков. Установка производилась с официального сайта проекта VirtualBox.

\subsubsection{Установка и загрузка гостевой операционной системы GNU/Linux Debian 12}

Для установки гостевой операционной системы использовался официальный минимальный образ \texttt{debian-12.11.0-arm64-netinst.iso}, скачанный с сайта проекта Debian. Архитектура ARM64 была выбрана в соответствии с платформой Apple M3, на которой запускалась виртуальная машина. Скриншот с сайта загрузки установочного образа приведён ниже:

\screenshot{Screenshot_2025-06-18_at_19.10.15.png}{Страница загрузки установочного образа Debian 12 для ARM64}

Установка операционной системы производилась в текстовом режиме.

Процесс установки включал следующие шаги:
\begin{itemize}
    \item В меню загрузчика выбрали пункт \texttt{Install}, чтобы начать установку в текстовом режиме.
    \item На этапе выбора языка интерфейса установки указали \texttt{English}.
    \item Для региона (локали) выбрали \texttt{United States}.
    \item Для раскладки клавиатуры оставили \texttt{American English}.
    \item Дождались автоматической настройки сети по DHCP.
    \item Задали имя хоста (\texttt{vbox}) для виртуальной машины.
    \item Поле доменного имени оставили пустым.
    \item Установили и подтвердили пароль пользователя \texttt{root}.
 \item Указали полное имя пользователя (\texttt{Ilya Semenov}).
    \item Указали имя пользователя для входа в систему (\texttt{semenov}).
    \item Задали пароль для нового пользователя и подтвердили его.
    \item Выбрали диск для разметки (виртуальный диск \texttt{sda} на 21.5 ГБ).
    \item Выбрали схему разметки \texttt{All files in one partition (recommended for new users)}.
    \item Подтвердили разметку и запись изменений на диск.
    \item На этапе настройки зеркала выбрали \texttt{deb.debian.org}.
    \item После завершения установки получили сообщение о завершении и готовы к перезагрузке.
\end{itemize}

Данные шаги отображены на скриншотах ниже.

\screenshot{VirtualBox_summer_p_semenov__01_07_2025_15_12_41.png}{Старт установки Debian, выбор текстового режима}
\screenshot{VirtualBox_summer_p_semenov__01_07_2025_15_13_13.png}{Выбор языка установки}
\screenshot{VirtualBox_summer_p_semenov__01_07_2025_15_13_28.png}{Выбор страны (локали)}
\screenshot{VirtualBox_summer_p_semenov__01_07_2025_15_13_37.png}{Выбор раскладки клавиатуры}
\screenshot{VirtualBox_summer_p_semenov__01_07_2025_15_13_52.png}{Настройка сети через DHCP}
\screenshot{VirtualBox_summer_p_semenov__01_07_2025_15_14_01.png}{Указание имени хоста}
\screenshot{VirtualBox_summer_p_semenov__01_07_2025_15_14_13.png}{Пропуск ввода доменного имени}
\screenshot{VirtualBox_summer_p_semenov__01_07_2025_15_14_28.png}{Установка пароля root}
\screenshot{VirtualBox_summer_p_semenov__01_07_2025_15_14_37.png}{Подтверждение пароля root}
\screenshot{VirtualBox_summer_p_semenov__01_07_2025_15_15_37.png}{Ввод полного имени пользователя}
\screenshot{VirtualBox_summer_p_semenov__01_07_2025_15_15_42.png}{Ввод имени пользователя для входа в систему}
\screenshot{VirtualBox_summer_p_semenov__01_07_2025_15_15_52.png}{Установка пароля пользователя}
\screenshot{VirtualBox_summer_p_semenov__01_07_2025_15_15_57.png}{Подтверждение пароля пользователя}
\screenshot{VirtualBox_summer_p_semenov__01_07_2025_15_16_24.png}{Выбор диска для установки}
\screenshot{VirtualBox_summer_p_semenov__01_07_2025_15_16_30.png}{Выбор схемы разметки}
\screenshot{VirtualBox_summer_p_semenov__01_07_2025_15_16_36.png}{Подтверждение записи разметки на диск}
\screenshot{VirtualBox_summer_p_semenov__01_07_2025_15_17_52.png}{Выбор зеркала репозитория Debian}
\screenshot{VirtualBox_summer_p_semenov__01_07_2025_15_20_06.png}{Завершение установки и готовность к перезагрузке}

\subsubsection{Настройка \flqqпроброса\frqq\ портов для подключения к гостевой ОС по SSH}

Для обеспечения доступа к гостевой системе по протоколу SSH необходимо настроить \flqqпроброс\frqq\ порта 22 из виртуальной машины на свободный порт хостовой операционной системы. Это позволяет подключаться к гостевой Debian из macOS с использованием локального IP-адреса и перенаправленного порта, что особенно удобно при работе через терминал.

В настройках сети виртуальной машины в VirtualBox был использован сетевой адаптер типа \texttt{NAT}. Далее во вкладке \texttt{Port Forwarding} было добавлено новое правило: хостовый порт 2222 (на \texttt{127.0.0.1}) перенаправляется на порт 22 гостевой системы.

Скриншот с настройкой правила \flqqпроброса\frqq\ порта приведён ниже:

\screenshot{Screenshot_2025-06-18_at_19.56.16.png}{Настройка \flqqпроброса\frqq\ 22 порта SSH на 2222 порт хоста в VirtualBox}

\subsubsection*{Приглашение к вводу имени пользователя после загрузки системы}

После завершения установки и перезагрузки гостевой операционной системы Debian 12 система автоматически запускается в текстовом терминале \texttt{tty1}, без графической оболочки. На экране появляется приглашение к вводу имени пользователя, оформленное в виде строки:

\screenshot{VirtualBox_summer_p_semenov_25_06_2025_22_24_04.png}{Экран приглашения к вводу логина после загрузки в терминале \texttt{tty1}}

Это подтверждает успешную загрузку системы, правильную настройку учётной записи и работу без графического интерфейса.

Ниже представлен успешный вход в систему под созданным ранее пользователем \texttt{semenov}:

\screenshot{VirtualBox_summer_p_semenov_18_06_2025_19_57_43.png}{Успешный вход в систему: отображение имени пользователя, имени хоста и приглашения командной строки}
