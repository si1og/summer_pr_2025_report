\section*{Приложение A. Заголовочный файл класса MyPlayer }
\addcontentsline{toc}{section}{Приложение A. Заголовочный файл класса MyPlayer}

\begin{lstlisting}[language=C++,caption={Заголовочный файл класса MyPlayer},label={lst:move_array},numbers=left]
#pragma once
#include "../core/game.hpp"
#include "../core/state.hpp"

namespace ttt::my_player {

using game::IPlayer;
using game::Point;
using game::Sign;
using game::State;

class MyPlayer : public IPlayer {
public:
    MyPlayer(const char *name);
    ~MyPlayer();

    void set_sign(Sign sign) override;
    Point make_move(const State &state) override;
    const char *get_name() const override;

private:
    char m_name[32];
    Sign m_sign;

    // Динамический массив ходов
    struct MoveArray {
        Point* data;
        int capacity;
        int size;
        MoveArray(int max_size);
        ~MoveArray();
        void clear();
        void push_back(const Point& p);
    };

    bool is_within_bounds(int x, int y, const State &state) const;
    bool has_neighbor(const State &state, int x, int y) const;
    void generate_moves(const State &state, MoveArray &moves) const;
    int evaluate(const State &state, Sign maximizer) const;
    int minimax(State &state, int depth, int alpha, int beta, bool maximizing,
                Sign player, Point &best_move, long long deadline) const;
};

} // namespace ttt::my_player
\end{lstlisting}