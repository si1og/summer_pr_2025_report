\vspace{1em}

\noindent
Алгоритм Minimax, применяя анализ только перспективных ходов, обеспечивает эффективный перебор вариантов и выбор оптимального хода для игрока X. Стратегия заключается в построении дерева возможных продолжений партии, где на каждом уровне оцениваются наиболее выгодные направления развития. Практика показывает, что такой подход позволяет быстро выявлять выигрышные или защитные ходы, минимизируя избыточные вычисления.


% \begin{center}
% \begin{tikzpicture}[
%   level distance=2.7cm,
%   sibling distance=4.1cm,
%   every node/.style={
%     font=\small,
%     align=center,
%     minimum width=1.8cm,
%     minimum height=1.1cm
%   },
%   solidnode/.style={
%     draw,
%     thick,
%     rectangle
%   },
%   dashednode/.style={
%     draw,
%     thick,
%     rectangle,
%     dashed
%   },
%   leafnode/.style={
%     draw,
%     thin,
%     rectangle,
%     font=\footnotesize,
%     minimum width=2.3cm,
%     minimum height=0.7cm
%   },
%   edge from parent/.style={
%     draw,
%     thick,
%     -latex
%   }
%   ]

% % Корень - позиция после хода 6
% \node[solidnode] (root) {Позиция \\ после хода 6}
%   child {node[solidnode,xshift=-3cm] (move1) {Ход X: (7,3)}
%     child {node[dashednode,xshift=1.5cm] (resp11) {O: (7,7)}
%       child {node[leafnode] {f=0}}
%     }
%     child {node[dashednode] (resp12) {O: (7,5)}
%       child {node[leafnode] {f=0}}
%     }
%     child {node[dashednode,xshift=-1.5cm] (resp13) {O: (8,3)}
%       child {node[leafnode] {f=-50}}
%     }
%   }
%   child {node[solidnode,xshift=1cm] (move2) {Ход X: (7,8)}
%     child {node[dashednode,xshift=1.5cm] (resp21) {O: (7,5)}
%       child {node[leafnode] {f=+80}}
%     }
%     child {node[dashednode] (resp22) {O: (8,8)}
%       child {node[leafnode] {f=-10}}
%     }
%     child {node[dashednode,xshift=-1.5cm] (resp23) {O: (6,8)}
%       child {node[leafnode] {f=-30}}
%     }
%   }
%   child {node[solidnode] (move3) {...}};

% % подписываем минимумы для каждой ветви
% \node[font=\footnotesize] at ($(move3.south) + (0,-0.5)$) {...};


% \node[font=\small\bfseries,red,below=1cm,xshift=-2cm] {Выбран ход (7,4) --- лучший результат минимакса};

% \end{tikzpicture}


% \end{center}
