\section*{Введение}
\addcontentsline{toc}{section}{Введение}

Данная курсовая работа посвящена разработке алгоритма для игры в классические «крестики-нолики» с расширенными правилами на поле размером $15 \times 15$.  В рамках этой работы требуется создать эффективного и сильного игрока-бота, способного конкурировать с заранее предоставленными игроками базового уровня.

Игра «крестики-нолики» представляет собой стратегическую игру для двух участников, где игроки поочерёдно ставят символы («X» или «O») в свободные клетки квадратного поля. Победа достигается при выстраивании непрерывной линии из пяти символов одного типа по горизонтали, вертикали или диагонали. Особенность данной работы состоит в том, что игра ведётся на большом игровом поле, что существенно увеличивает сложность и делает неприменимым прямой перебор всех возможных ходов. Это требует от разработчика реализации эффективных алгоритмов выбора хода, способных быстро оценивать перспективные варианты и принимать оптимальные решения в условиях ограниченного времени (50–100 миллисекунд на ход).

В рамках данной работы для построения стратегии выбора ходов выбран классический алгоритм Minimax с альфа-бета отсечением и оптимизациями по сокращению пространства поиска. Данный подход позволяет эффективно находить оптимальные ходы, рассматривая только действительно перспективные варианты развития позиции, и строить игру, максимально приближенную к теоретически оптимальной.

В последующих разделах отчёта будут представлены подробная постановка задачи, математическое описание используемого алгоритма, описание особенностей реализации, а также результаты тестирования разработанного решения.