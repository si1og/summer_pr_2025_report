\subsection{Перспективные ходы: хранение, вычисление, перебор}

В задаче эффективной реализации алгоритма Minimax для игры на большом поле (например, $15 \times 15$) крайне важно не перебирать все пустые клетки, а лишь так называемые \textbf{перспективные ходы} — то есть только те, которые действительно могут изменить ситуацию на поле.

\subsubsection*{1. Хранение перспективных ходов}

В программе перспективные ходы обычно хранятся в виде \textbf{массива пар целых чисел}:

\begin{itemize}
    \item Например, массив \verb|moves[]|.
    \item При расчёте каждый перспективный ход добавляется в этот массив, дубли не допускаются.
\end{itemize}

\subsubsection*{2. Как рассчитываются перспективные ходы}

\textbf{Алгоритм:}
\begin{enumerate}
    \item Просмотреть все занятые клетки (где уже стоят X или O).
    \item Для каждой занятой клетки перебрать всех 8 соседей (по горизонтали, вертикали и диагонали).
    \item Если соседняя клетка пуста и ещё не добавлена в массив перспективных — добавить её.
    \item Если поле пустое — добавить в перспективные центр поля (стартовый ход).
\end{enumerate}

\vspace{0.5em}

\textbf{Пример на псевдокоде:}

\begin{verbatim}
Переменная moves = пустой массив

Для всех x от 0 до N-1:
    Для всех y от 0 до N-1:
        Если field[x][y] занята (X или O):
            Для dx от -1 до 1:
                Для dy от -1 до 1:
                    Если dx = 0 и dy = 0, то пропустить
                    nx = x + dx; ny = y + dy
                    Если (nx, ny) внутри поля и field[nx][ny] пусто:
                        Если (nx, ny) ещё не в moves:
                            Добавить (nx, ny) в moves

Если moves пуст:
    // если поле пустое, начать с центра
    Добавить (N//2, N//2) в moves 
\end{verbatim}

\begin{itemize}
    \item Так в \verb|moves| окажутся только пустые клетки, расположенные непосредственно рядом с занятыми, то есть те, где ход имеет смысл.
    \item Обычно число перспективных ходов на практике не превышает 20–40 даже на большом поле, что позволяет делать Minimax эффективным.
\end{itemize}

\subsubsection*{3. Перебор перспективных ходов в Minimax}

При запуске Minimax бот перебирает \textbf{только} перспективные ходы:

\begin{itemize}
    \item Для каждой клетки из массива \verb|moves|:
        \begin{itemize}
            \item Симулирует свой ход в эту клетку (делает временно X или O).
            \item Рекурсивно вызывает Minimax для полученного нового поля.
            \item Сохраняет оценку позиции.
        \end{itemize}
    \item Из всех результатов выбирает ход с максимальной (или минимальной — для соперника) оценкой.
    \item В ходе поиска работает \textbf{альфа-бета отсечение} — если текущая ветка гарантированно хуже уже найденной, она не исследуется дальше.
\end{itemize}

\vspace{0.5em}

\textit{Перспективные ходы позволяют резко сузить область поиска Minimax, что делает возможным глубокий перебор даже на больших полях и увеличивает скорость принятия решения.}
