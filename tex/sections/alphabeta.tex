\subsection{Принцип альфа-бета отсечения}

Альфа-бета отсечение (\textit{alpha-beta pruning})~— это усовершенствование метода полного перебора Minimax, позволяющее существенно сократить количество рассматриваемых позиций в дереве поиска без потери качества результата.

В процессе рекурсивного перебора поддерживаются два параметра:
\begin{itemize}
    \item $\alpha$~— наилучшая (максимальная) оценка, которую гарантированно может получить максимизатор ($X$) на текущем или любом из предыдущих уровней поиска,
    \item $\beta$~— наилучшая (минимальная) оценка, которую может гарантировать себе минимизатор ($O$) на текущем или любом из предыдущих уровней.
\end{itemize}

При просмотре очередного ответа соперника выполняется проверка:  
если $\beta \leq \alpha$, то дальнейшее рассмотрение ходов в этой ветви не имеет смысла, так как минимизатор уже сможет гарантировать себе результат не хуже найденного, а максимизатор — не лучше найденного, соответственно итог выбора не изменится.  
Такие ветви \textbf{отсекаются} (pruned), что приводит к экспоненциальному уменьшению числа вызовов функции Minimax.

\vspace{1em}

\textbf{Пояснение:}  
Альфа-бета отсечение не влияет на итоговый выбранный ход — оно лишь позволяет не рассматривать явно заведомо проигрышные/ненужные поддеревья поиска, повышая эффективность работы алгоритма, особенно на больших игровых полях.
