\section*{Заключение}
\addcontentsline{toc}{section}{Заключение}

В ходе выполнения данной работы был разработан и реализован алгоритмический игрок-бот для обобщённой игры «крестики-нолики» на большом поле $15 \times 15$ с победой по линии из пяти символов. Для построения стратегии выбора ходов использовался метод Minimax с альфа-бета отсечением, а также оптимизации по сокращению пространства перебора (генерация только перспективных ходов).

Были написаны основные компоненты: класс игрока (\texttt{MyPlayer}) с полным интерфейсом для интеграции в систему тестирования, а также вспомогательные структуры и функции (эвристика оценки позиции, динамический массив ходов, генерация перспективных ходов и др.). Суммарно, реализация игрока составила около 200-250 строк кода (без учёта комментариев, тестов и стандартных заголовков).

В результате автоматического тестирования бот уверенно выигрывает у базового (простого) соперника, показывая корректную работу, высокую скорость принятия решений (среднее время на ход около 20--25 мс при глубине рекурсии 2) и стабильные результаты (80--85 побед из 100 партий). Против продвинутого соперника бот пока проигрывает все партии, что связано с ограничениями по глубине перебора и простотой используемой эвристики. При увеличении глубины поиска до 3 бот начинает не успевать укладываться во временной лимит (50--100~мс), что ведёт к снижению качества игры даже против слабых соперников.

В процессе работы были приобретены навыки проектирования эффективных игровых агентов, работы с поисковыми алгоритмами и анализа временной производительности. Основными трудозатратами стали: проектирование структуры класса и интерфейса (около 4 часов), реализация и отладка Minimax с эвристикой (около 7 часов), написание вспомогательных методов и тестирование (около 3 часов), оформление отчёта (10 часов). В целом, на выполнение всей работы потребовалось примерно 24 часа.

В качестве направлений для дальнейшего развития можно предложить:
\begin{itemize}
    \item усовершенствование эвристической функции (например, учитывать типы блоков, «открытые» и «закрытые» линии);
    \item внедрение сортировки перспективных ходов для более эффективного альфа-бета отсечения;
    \item реализация итеративного углубления для полного использования лимита времени;
    \item кеширование оценок для избежания повторных расчётов.
\end{itemize}
