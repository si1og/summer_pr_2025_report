\section{Результаты тестирования}

Для оценки качества работы разработанного бота были проведены автоматические тесты с использованием стандартных утилит из репозитория \texttt{tictactoe-course}. Тестирование проводилось на нескольких сценариях:

\begin{enumerate}
    \item Игра против простого базового игрока (\texttt{BaselineEasy}), глубина рекурсии Minimax~--- 2.
    \item Игра против сложного базового игрока (\texttt{BaselineHard}), глубина рекурсии Minimax~--- 2.
    \item Игра против простого игрока с глубиной рекурсии 3.
    \item Игра против самого себя (для оценки корректности и времени).
\end{enumerate}

\vspace{0.5em}
Ниже приведены типовые результаты тестирования с комментариями:

\subsection{Тест против простого игрока (\texttt{BaselineEasy}), глубина рекурсии 2}

\begin{lstlisting}[caption={Тест против BaselineEasy, глубина 2},label={lst:test_easy2},numbers=none]
Testing MyPlayer vs baseline easy player
MyPlayer wins: 84
BaselineEasy wins: 15
draws: 1
errors: 0

MyPlayer play time:
 - move time (ms): 23.8649
 - event time (ms): 0.00107065

BaselineEasy play time:
 - move time (ms): 0.0119827
 - event time (ms): 0.0111717

game process average time: 11.9594 (ms)
\end{lstlisting}

При глубине рекурсии 2 бот выигрывает у простого игрока подавляющее большинство партий. Среднее время расчёта одного хода~--- порядка 24~мс, что значительно меньше лимита (100~мс).

\subsection{Тест против сложного игрока (\texttt{BaselineHard}), глубина рекурсии 2}

\begin{lstlisting}[caption={Тест против BaselineHard, глубина 2},label={lst:test_hard2},numbers=none]
Testing MyPlayer vs baseline hard player
MyPlayer wins: 0
BaselineHard wins: 100
draws: 0
errors: 0

MyPlayer play time:
 - move time (ms): 22.7835
 - event time (ms): 0.00119048

BaselineHard play time:
 - move time (ms): 0.0140847
 - event time (ms): 0.0121181

game process average time: 11.421 (ms)
\end{lstlisting}

При стандартной глубине (2) бот пока уступает полностью сложному игроку, что объясняется высокой эффективностью стратегии соперника и малой глубиной поиска нашего бота.

\subsection{Тест против простого игрока, глубина рекурсии 3}

\begin{lstlisting}[caption={Тест против BaselineEasy, глубина 3},label={lst:test_easy3},numbers=none]
Testing MyPlayer vs baseline easy player
MyPlayer wins: 74
BaselineHard wins: 25
draws: 1
errors: 0

MyPlayer play time:
 - move time (ms): 84.3088
 - event time (ms): 0.00120689

BaselineHard play time:
 - move time (ms): 0.0143187
 - event time (ms): 0.0131544

game process average time: 42.1859 (ms)
\end{lstlisting}

При увеличении глубины рекурсии до 3 бот начинает тратить существенно больше времени на расчёт ходов. Несмотря на то, что соперник остаётся простым, качество игры неожиданно падает, и число поражений увеличивается. Это связано с тем, что из-за ограниченного лимита времени (100~мс на ход) алгоритм не всегда успевает просчитать все варианты до заданной глубины. В ряде случаев перебор прерывается досрочно, и решение принимается на основе неполного анализа позиции, что приводит к ошибочным или невыгодным ходам. Данный эффект подтверждает важность разумного баланса между глубиной Minimax и ограничением времени.

\subsection{Тест бота против самого себя}

\begin{lstlisting}[caption={Тест MyPlayer против MyPlayer},label={lst:test_self},numbers=none]
Testing MyPlayer vs MyPlayer
MyPlayer wins: 100
MyPlayer wins: 0
draws: 0
errors: 0

MyPlayer play time:
 - move time (ms): 8.52504
 - event time (ms): 0.00119167

MyPlayer play time:
 - move time (ms): 13.1122
 - event time (ms): 0.000942222

game process average time: 10.8284 (ms)
\end{lstlisting}

Внутренние тесты на корректность (бот против самого себя) подтверждают стабильную работу алгоритма: все партии завершаются корректно, среднее время на ход~--- $8$--$13$~мс.

\subsection{Выводы по результатам тестирования}

Проведённые автоматические тесты показывают, что реализованный бот демонстрирует устойчивое преимущество против простого (BaselineEasy) соперника, выигрывая подавляющее большинство партий при глубине рекурсии 2. При этом среднее время принятия решения по ходу укладывается в установленный лимит (50--100~мс), что подтверждает эффективность выбранных оптимизаций — в первую очередь, генерации перспективных ходов и использования альфа-бета отсечения.

Против более сложного игрока (BaselineHard) бот уже проигрывает все партии, что связано с ограниченной глубиной поиска и простотой используемой эвристики. При увеличении глубины поиска до 3 заметно возрастает время работы алгоритма, и бот всё чаще вынужден принимать решения на неполном анализе из-за ограничения по времени — что негативно сказывается на результате даже против простого соперника.

Таким образом, выбранный подход обеспечивает корректную работу и конкурентоспособность на уровне простых и средних соперников, но для дальнейшего повышения силы потребуется совершенствовать эвристику и реализовать дополнительные оптимизации перебора (например, более умное ранжирование перспективных ходов, применение методов поиска с итеративным углублением и пр.).
