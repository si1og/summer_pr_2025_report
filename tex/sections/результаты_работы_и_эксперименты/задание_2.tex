\subsection{Сборка, тестирование и исследование зависимостей программы, осуществляющей работу с базой данных через псевдографический интерфейс}

\subsubsection{Сборка программы}

Для сборки программы был подготовлен файл \texttt{CMakeLists.txt}, который задаёт проект на языке C, стандарт C99 и настраивает поиск необходимых библиотек --- \texttt{ncursesw} (через \texttt{PkgConfig}) и \texttt{zlib}. После их нахождения добавляется исполняемый файл \texttt{app}, которому передаются пути к заголовочным файлам и библиотеки для линковки.

Редактирование данного файла производилось в редакторе \texttt{nano}, что показано на рисунке ниже:

\screenshot{Screenshot_2025-07-01_at_01.04.47.png}{Редактирование CMakeLists.txt в редакторе nano}

После этого был создан каталог \texttt{build}, в который были сгенерированы файлы сборочной системы командой \texttt{cmake ..}, и выполнена сама сборка проекта с помощью утилиты \texttt{make}. На выходе получен исполняемый файл \texttt{app}, который связывается с найденными библиотеками.

Ход выполнения этих команд и процесс сборки приведён на скриншоте \ref{fig:Screenshot_2025-07-01_at_01.12.42.png}.

\screenshot{Screenshot_2025-07-01_at_01.12.42.png}{Конфигурация и сборка проекта с использованием CMake и Make}

Таким образом, проект был успешно собран с подключением библиотек \texttt{ncursesw} для реализации псевдографического интерфейса и \texttt{zlib} для вычисления контрольных сумм. Исполняемый файл готов к тестированию и дальнейшему исследованию его зависимостей через команду \texttt{ldd}.

\subsubsection{Тестирование программы}

Программа была протестирована вручную: с помощью псевдографического интерфейса вводились новые записи, выполнялся просмотр заголовка и содержимого базы данных, поиск по знаку зодиака и году рождения, а также добавление новых записей. Результаты работы программы представлены на приведённых скриншотах, которые демонстрируют все основные этапы и корректное функционирование программы.

\screenshot{Screenshot_2025-07-01_at_01.43.35.png}{Запуск программы. Отсутствие базы данных и запрос на ввод количества записей.}
\screenshot{Screenshot_2025-07-01_at_01.44.06.png}{Ввод первой записи: фамилии, знака зодиака и года рождения.}
\screenshot{Screenshot_2025-07-01_at_01.44.12.png}{Главное меню программы с выбором действий.}
\screenshot{Screenshot_2025-07-01_at_01.44.15.png}{Просмотр заголовка базы данных с указанием количества записей и контрольной суммы.}
\screenshot{Screenshot_2025-07-01_at_01.44.20.png}{Просмотр всех записей в базе данных в табличном виде.}
\screenshot{Screenshot_2025-07-01_at_01.44.33.png}{Запрос параметров для поиска по знаку зодиака и году рождения.}
\screenshot{Screenshot_2025-07-01_at_01.44.35.png}{Результаты поиска записей по заданным критериям.}
\screenshot{Screenshot_2025-07-01_at_01.46.06.png}{Добавление новой записи в базу данных.}
\screenshot{Screenshot_2025-07-01_at_01.46.13.png}{Просмотр обновлённой базы данных с двумя записями.}
\screenshot{Screenshot_2025-07-01_at_01.46.31.png}{Просмотр обновлённого заголовка базы данных с увеличенным количеством записей и новой контрольной суммой.}
\screenshot{Screenshot_2025-07-01_at_01.47.33.png}{Удаление файла базы данных перед повторным тестированием.}
\screenshot{Screenshot_2025-07-01_at_01.47.45.png}{Запуск программы без базы данных и ввод некорректного количества записей (-1).}
\screenshot{Screenshot_2025-07-01_at_01.47.48.png}{Сообщение о недопустимом количестве записей и предложение повторить ввод.}
\screenshot{Screenshot_2025-07-01_at_01.49.15.png}{Повторный запуск программы с вводом 20 записей.}
\screenshot{Screenshot_2025-07-01_at_01.50.06.png}{Ввод записи с некорректным годом рождения вне допустимого диапазона.}
\screenshot{Screenshot_2025-07-01_at_01.50.09.png}{Сообщение о недопустимом году рождения и повторный запрос.}
\screenshot{Screenshot_2025-07-01_at_01.50.58.png}{Повторный воод года рождения.}
\screenshot{Screenshot_2025-07-01_at_01.51.11.png}{Сообщение о пустом знаке зодиака и повторный запрос.}
\screenshot{Screenshot_2025-07-01_at_01.55.18.png}{Просмотр заголовка базы данных после ввода записей.}
\screenshot{Screenshot_2025-07-01_at_01.55.29.png}{Просмотр всех записей базы данных в табличном виде после ввода 20 элементов.}
\screenshot{Screenshot_2025-07-01_at_01.55.32.png}{Просмотр вротой страницы всех записей базы данных в табличном виде после ввода 20 элементов.}
\screenshot{Screenshot_2025-07-01_at_01.56.16.png}{Поиск среди множества записей по знаку зодиака \texttt{f} и году \texttt{2000}}
\screenshot{Screenshot_2025-07-01_at_01.56.59.png}{Поиск заведомо некорректными данными}
\screenshot{Screenshot_2025-07-01_at_01.57.51.png}{Добавление записи на имя \texttt{petrov v v}}
\screenshot{Screenshot_2025-07-01_at_01.58.00.png}{Просмотр содержимого второй страницы обновлённой базы данных.}
\screenshot{Screenshot_2025-07-01_at_01.58.10.png}{Просмотр заголовка новой базы данных после создания --- указано количество записей и контрольная сумма.}
